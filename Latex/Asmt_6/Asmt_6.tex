\documentclass{report}

\usepackage{pgfplots}
\pgfplotsset{width=5.5in,compat=1.10}

 \begin{document}
 
\begin{center}
\textbf{1. Linear Regression (75 points)}
\end{center}

\begin{flushleft}
 We will find coefficients A to estimate $X*A \approx Y$, using the provided datasets X and Y. We will compare two approaches least squares and ridge regression.
 
 Least Squares: Set A = inverse(X’ * X)*X’*Y
 
 Ridge Regression: Set As = inverse(X’*X + sˆ2*eye(15))*X’*Y
\end{flushleft}

 \begin{flushleft}
A(30 points) Solve for the coefficients A (or As) using Least Squares and Ridge Regression with s = {1, 5, 10, 15, 20, 25, 30} (i.e. s will take on one of those 7 values each time you try, say obtaining A05 for s = 5). For each set of coefficients, report the error in the estimate \^{Y} of Y as norm(Y - X*A,2).
\end{flushleft}

 \begin{flushleft}
Error A = 605.6547

Error AS 1 = 605.6549

Error AS 5 = 605.7708

Error AS 10 = 607.1568

Error AS 15 = 611.3054

Error AS 20 = 618.4469

Error AS 25 = 627.9439

Error AS 30 = 639.1680
\end{flushleft}

 \begin{flushleft}
B (30 points): Create three row-subsets of X and Y
\end{flushleft}

 \begin{flushleft}
X1 = X(1:66,:) and Y1 = Y(1:66)

Error A = 483.0236

Error AS 1= 482.7316

Error AS 5 = 476.4220

Error AS 10 = 462.9247

Error AS 15 = 451.7693

Error AS 20 = 446.5091

Error AS 25 = 446.8578

Error AS 30= 451.7081
\end{flushleft}

\begin{flushleft}
X2 = X(34:100,:) and Y2 = Y(34:100)

Error A = 434.7992

Error AS 1 = 434.5322

Error AS 5 = 428.8565

Error AS 10 = 417.0631

Error AS 15 = 406.8163

Error AS 20 = 400.1875

Error AS 25 = 397.0574

Error AS 30 = 397.1167
\end{flushleft}

\begin{flushleft}
X3 = [X(1:33,:); X(67:100,:)] and Y3 = [Y(1:33); Y(67:100)]

Error A = 383.0319

Error AS 1 = 382.7135

Error AS 5 = 376.3331

Error AS 10 = 365.6099

Error AS 15 = 359.6731

Error AS 20 = 358.5897

Error AS 25 = 360.6872

Error AS 30 = 365.0511
\end{flushleft}

\begin{flushleft}
C (15 points): Which approach works best (averaging the results from the three subsets): Least Squares, or for which value of s using Ridge Regression?
\end{flushleft}

\begin{flushleft}
Error A = 433.6182

Error AS 1 = 433.3258

Error AS 5 = 427.2039

Error AS 10 = 415.1992

Error AS 15 = 406.0862

Error AS 20 = 401.7621

Error AS 25 = 401.5342

Error AS 30 = 404.6253
\end{flushleft}

\begin{flushleft}
Ridge Regression with s = 25  seems to be the best because the lowest value out of the average is Error AS 25 = 401.5342. The lowest error is the best
\end{flushleft}

\begin{center}
\textbf{2. Orthogonal Matching Pursuit (25 points)}
\end{center}

\begin{flushleft}
Consider a linear equation W = M*S where M is a measurement matrix filled with random values  (although now that they are there, they are no longer random), and W is the output of the sparse signal S when measured by M.
Use Orthogonal Matching Pursuit (as described in the notes as Algorithm 18.2.1) to recover the non-zero entries from S. Record the order in which you find each entry and the residual vector after each step.
\end{flushleft}

\begin{flushleft}
The non-zero entries are given by the given indexes below and the residual vectors after each step are also given below.
$\newline$

r = [-3     0     0    -3     5     1     1     2     0    -2    -2    -2    -4     1     0     2     1     0     3    -1    -2    -1    -2     1    -2    -2     2    -1     0     3     4     2     1    -1     1     1    -4     2     0     1     0    -1    -1    -3     1    -1     1     2    -2     2     2     0    -1    -2     0     0    -1    -3     0     2    -2     0    -2     1    -1     0    -4    -3    -1     0     3     0     0     0     4     1     3    -3    -1    -3
$\newline$

index = 43
$\newline$

r = [-2     1     1    -2     4     1     2     1     1    -2    -1    -2    -3     2     1     1     0     0     3    -1    -1     0    -1     1    -2    -1     3    -1     0     2     3     1     1     0     0     0    -3     2     1     0     0     0     0    -2     0     0     1     2    -2     1     1    -1    -1    -2    -1    -1     0    -2    -1     2    -2     0    -1     0    -2     0    -3    -2    -2     0     2     0     1    -1     3     1     3    -3    -1    -2]
$\newline$

index = 66
$\newline$

r = [-1     0     1    -1     3     1     1     1     1    -2    -2    -1    -2     1     2     1    -1     0     2     0    -1     0     0     0    -2    -1     3    -2    -1     2     2     1     0     1     0     0    -2     1     0    -1    -1     0    -1    -1     1     1     0     2    -1     0     0    -1     0    -1     0    -1     1    -1     0     2    -1    -1     0     1    -1    -1     -2    -2    -1     0     2     0     0     0     2     1     2    -2    -2    -1]
$\newline$

index =16
$\newline$

r = [0    -1     0    -1     2     2     1     1     1    -1    -1     0    -2     1     1     0     0     0     1     0     0     1    -1    -1    -1    -1     2    -2    -1     1     2     1     1     1    -1     0    -1     1    -1     0    -1    -1     0    -2    1     0     0     1     0     0     0     0     1     0    -1     0     1    -1     1     1    -1    -1     1     1     0    -1    -1    -1     0    -1     1     0     0     0     2     2     2    -1    -1    -1]
$\newline$

index = 65
$\newline$

r = [0    -1     1    -1     1     1     0     1     1     0     0     0    -1     1     0     1     0     0     1     1    -1     1     0     0     0    -1     1    -1     0     1     1     1     0     1     0     0    -1     0    -1     0     0    -1     1    -1     0     1     0     0    -1     1    -1     0     0    -1     0    -1     0     0     1     0    -1     0     0     1     1     0     -1     0    -1    -1     0     0     1     0     1     1     1    -1     0    -1]
$\newline$

index = 89
$\newline$

r = [0     0     0     0     0     0     0     0     0     0     0     0     0     0     0     0     0     0     0     0     0     0     0     0     0     0     0     0     0     0     0     0     0     0     0     0     0     0     0     0     0     0     0     0     0     0     0     0     0     0     0     0     0     0     0     0     0     0     0     0     0     0     0     0     0     0     0     0     0     0     0     0     0     0     0     0     0     0     0     0]
\end{flushleft}

 \begin{flushleft}
The Assignment was done in Matlab where index start at 1 instead of 0. The elements at index 43, 66, 16, 65, and 89 are the non-zero entries from S.
\end{flushleft}

\end{document}